\documentclass[10pt,a4paper]{article}
\usepackage[utf8x]{inputenc}
\usepackage{ucs}
\usepackage[english]{babel}
\usepackage{amsmath}
\usepackage{amsfonts}
\usepackage{amssymb}
\usepackage{makeidx}
\usepackage{graphicx}
\usepackage{lmodern}
\usepackage{kpfonts}
\usepackage{float}

\usepackage[left=2cm,right=2cm,top=2cm,bottom=2cm]{geometry}
%\begin{document}

\headsep17mm
\topmargin-1cm
\hoffset -1.5cm \voffset -1cm \textwidth 17cm \oddsidemargin
1.5cm \evensidemargin 1.5cm \textheight 22.5cm

\newcommand{\coef}[2]{\left( \begin{array}{c} #1 \\ #2 \end{array}\right)}
\newcommand{\atr}{\hspace{-0,12cm}}
%\newcommand{\atrs}{\hspace{-0,1cm}}

\newcounter{cuent}
\newcommand{\proba}[1]{\stepcounter{cuent}{\alph{cuent})\quad}
\displaystyle#1\qquad}
\newcommand{\cuento}{\setcounter{cuent}{0}}

\begin{document}
\noindent {\bf Universidad de la Rep\'{u}blica} \hfill
          {\bf C\'{a}lculo diferencial e integral en una variable} \\
{\bf Facultad de Ingenier\'{\i}a - IMERL} \hfill {\bf Primer semestre 2018}

\vspace{0,3cm}

\begin{center}
{\bf \Large Soluciones seleccionadas, Semana 01}
\end{center}

\vspace{0,3cm}

\section*{2.1.j}

\emph{Calcular las raices de $P(x) = x^4-x^2-2$}


\noindent
Los polinomios de esta forma (grado 4 con coeficientes en la primer y tercer
potencia nulos) pueden transformarse y resolverse como si fuesen de segundo
grado.
Consideramos el cambio de variable
$u := x^2$. El polinomio queda $P(x) = u^2-u-2$,
cuyas raices (en función de $u$), aplicando la f\'ormula
de Bhaskara son:

\begin{equation*}
  \begin{split}
    &u_1 = \frac{-1 + \sqrt{(-1)^2-4\cdot 1 \cdot (-2)}}{2 \cdot 1} = 1\\
    &u_2 = \frac{-1 - \sqrt{(-1)^2-4\cdot 1 \cdot (-2)}}{2 \cdot 1} = -2
  \end{split}
\end{equation*}

\noindent
Luego, encontramos los $x$ para los cuales $u$ toma esos valores.
Es decir, resolvemos las ecuaciones $x^2 = u_i$
para obtener las raices de $P(x)$.
De $x^2=1$ obtenemos $x_1=-1$, $x_2 = 1$. De $x^2=-2$ no se obtienen nuevas
raices reales. Si estamos trabajando en el cuerpo de los complejos, obtenemos
$x_3 = \sqrt{2} i$, y $x_4 = -\sqrt{2} i$

\section*{3.2.j}
Determinar para qu\'e valores de $x$ es cierta la siguiente inecuaci\'on:
\begin{equation*}
  \vert 2x - 5 \vert  <  \vert 3x + 4 \vert
\end{equation*}

Observamos que la definici\'on de valor absoluto, depende del signo

\begin{equation*}
  \vert x \vert =
  \begin{cases}
    x & \mbox{si } x\geq 0 \\
   -x & \mbox{si } x < 0
  \end{cases}
\end{equation*}



\end{document}
