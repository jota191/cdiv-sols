\documentclass[10pt,a4paper]{article}
\usepackage[utf8x]{inputenc}
\usepackage{ucs}
\usepackage[english]{babel}
\usepackage{amsmath}
\usepackage{amsfonts}
\usepackage{amssymb}
\usepackage{makeidx}
\usepackage{graphicx}
\usepackage{lmodern}
\usepackage{kpfonts}
\usepackage{float}
\usepackage{enumitem}

\usepackage[left=2cm,right=2cm,top=2cm,bottom=2cm]{geometry}

\usepackage{titlesec}

\renewcommand{\o}{\circ}
\newcommand{\qed}{\blacksquare}

% elimina newline despues de \section:
\titleformat{\section}[runin]
{\normalfont\large\bfseries}{\thesubsection}{1em}{}      


\headsep17mm
\topmargin-1cm
\hoffset -1.5cm \voffset -1cm \textwidth 17cm \oddsidemargin
1.5cm \evensidemargin 1.5cm \textheight 22.5cm




\begin{document}

\vspace{0,3cm}

\begin{center}
{\bf \Large Resoluciones seleccionadas, semana 04}
\end{center}


\vspace{0,3cm}

\section*{4.1.9}\emph{}%TODO escribir letra

\noindent
Recordemos que las siguientes afirmaciones son equivalentes,
para un conjunto $A \subset \mathbb{R}$:

\begin{enumerate}[label=S-\Roman*)]
\item $\alpha$ es el supremo de $A$.
\item $\alpha$ es cota superior de $A$ y
  $\forall \delta > 0, \: \exists a \in A \:$ tal que
  $ a \in (\alpha -\delta, \alpha]$ ($ \alpha - a < \delta $.).
\end{enumerate}

\noindent
An\'alogamente son equivalentes:

\begin{enumerate}[label=I-\Roman*)]
\item $\alpha$ es el \'infimo de $A$.
\item $\alpha$ es cota inferior de $A$ y
  $\forall \delta > 0, \: \exists s \in A \:$ tal que
  $ a \in [\alpha, \alpha + \delta)$, ($a - \alpha < \delta$).
\end{enumerate}

\noindent
Usualmente ocurre que es m\'as c\'omodo demostrar que un real es supremo o
\'infimo usando estas condiciones equivalentes a la definici\'on,
(que se prueban, son teoremas).

\begin{enumerate}
\item
  \begin{enumerate}
  \item
    Obs\'ervese que los valores a estudiar est\'an bien definidos. Si $B$
    est\'a acotado tiene supremo e \'infimo bien definidos (por Axioma de
    Completitud).
    Sea $c$ es una cota inferior de $B$, por definición
    $\forall b \in B, c \leq b$. Pero $\forall a \in A$ tenemos que $a\in B$
    (por la inclusi\'on $A \subset B$),
    entonces $c \leq a$ y $c$ es cota inferior de $A$.
    Análogamente razonamos con los supremos. Por tanto $A$ también está acotado
    y existen $sup(A)$ e $inf(A)$ por Axioma de Completitud.
    Pero si toda cota de $B$ es cota de $A$ en particular
    $inf(B)$ es cota inferior de $A$, por tanto como $inf(A)$ es
    la mayor de las cotas inferiores, $inf(B) \leq inf(A)$.
    An\'alogamente deducimos que $sup(A) \leq sup(B)$.
    Finalmente para demostrar que $inf(A) \leq \sup(A)$ alcanza con considerar
    $a_0 \in A$ (recordemos que $A$ es no vac\'io por lo que existe $a_0$).
    Como $inf(A)$ es cota inferior, $inf(A) \leq a_0$.
    An\'alogamente $a_0 \leq sup(A)$, y por transitividad tenemos
    lo que quer\'iamos.
  \item
    $A= [0,\frac{1}{2}]$, $B = [0,1]$.
  \end{enumerate}
\item
  \begin{enumerate}
  \item
    Por absurdo, supongamos que $sup(A) > inf(B)$. La intuici\'on es que,
    podemos encontrar un de $a \in A$ tan cercano a
    $sup (A)$ como queramos,
    y un $b\in B$ tan cercano a $inf (B)$ como queramos. De forma tal
    que $b<a$, contradiciendo la hip\'otesis.
    Formalmente, sea $a \in A$ tal que $sup(A)-a < \frac{sup(A)-inf(B)}{2}$
    (Por S-II con $\delta = \frac{sup(A)-inf(B)}{2}$).
    Análogamente consideramos $b\in B$ tal que
    $b - inf(B)< \frac{sup(A)-inf(B)}{2}$.
    Sumando miembro a miembro y usando las propiedades de orden:

    $$ \frac{sup(A)-inf(B)}{2} + \frac{sup(A)-inf(B)}{2} =
    sup(A)-inf(B) >sup(A) - a + b - inf(B) \geq sup(A) - inf(B) + b - a
    $$
    \noindent
    De donde $0 > b-a$ y por tanto $b<a$. Absurdo.
    El absurdo viene de suponer que $sup(A)> inf(B)$, por tanto
    $sup(A) \leq \inf(B)$ como quer\'iamos.

  \item
    $A = [0,1]$, $B=(1,2]$.
  \end{enumerate}
\newpage
\item
  \begin{enumerate}
  \item Vamos a demostrar que $inf(A) + inf(B)$ es el \'infimo de $A+B$,
    usando la condici\'on I-II. El caso del supremo es an\'alogo.
    Primero probemos que $inf(A)+inf(B)$ es cota inferior de $A+B$.
    Como $inf(A)$ es cota inferior de $A$, $\forall a \in A$ tenemos que
    $inf(A) \leq a$ (1). An\'alogamente como $inf(B)$ es cota inferior de $B$,
    $\forall b \in B$, $inf(B) \leq b$ (2).
    Sea $x \in B$ se cumple $x = a_x+b_x$ para alg\'un $a_x\in A$, $b_x \in B$.
    Por tanto 
    por (1), en particular\footnote{N\'otese que (1) se cumple para {\bf todo} 
      $a \in A$, por tanto puedo instanciar el teorema para cualquier
      elemento en particular (por ejemplo$a_x$) }
    $\inf(A) \leq a_x$ y an\'alogamente
    $\inf(B) \leq b_x$. Sumando miembro a miembro
    $\inf(A)+ inf(B) \leq a_x + b_x = x$. Como $x$ era arbitrario,
    deducimos que $\forall x \in A+B, \: \inf(A)+ inf(B) \leq x$ por lo que
    $inf(A)+inf(B)$ es una cota inferior, como quer\'iamos.
    \noindent
    Por otra parte, para los \'infimos por hip\'otesis
    se cumple:

    $$\forall \delta>0, \: \exists a \in A \: \text{tal que }
    \: a_{\delta} - inf(A) < \frac{\delta}{2} $$

    Nótese que elegimos $\frac{\delta}{2}$ para tener $\delta$
    en la definici\'on
    final (para $\inf(A+B)$). Podríamos escribir esto con
    cualquier constante, dado que la desigualdad se cumple para cualquier
    $\delta>0$.

    Tambi\'en:

    $$\forall \delta>0, \: \exists b \in B \: \text{ tal que }
    \: b_{\delta} - inf(B) < \frac{\delta}{2} $$
    
    Dado $\delta$, construimos $x \in A+B$ eligiendo un par $a_{\delta}$
    y $b_{\delta}$, siendo $ x= a_{\delta}+ b_{\delta}$. Sumando miembro a miembro:

    $$
    a_{\delta} - inf(A) + b_{\delta} - inf(B)
    = a_{\delta} + b_{\delta} - (inf(A) + inf (B))
    < \frac{\delta}{2} + \frac{\delta}{2}= \delta
    $$

    Entonces finalmente:
    $$\forall \delta > 0, \: \exists x \in A+B \: \text{ tal que }
    \: x_{\delta} - (inf(A)+inf(B)) < \delta $$
    Demostramos entonces (condici\'on I-II) que $inf(A)+inf(B) = inf (A+B)$.
    
  \item
    Con $B=(0,1]$ se satisface $(0,2] + (0,1] = (0,3]$.
    $inf(C) = inf(A) = inf(B) = 0$ t $sup(C) = 3$, $sup(A) = 2$, $sup(B)=1$.
            Trivialmente se cumplen las igualdades de la parte anterior.
  \end{enumerate}
\item
  \begin{enumerate}
  \item
    Usamos la condici\'on S-II.
    Para demostrar que $\alpha sup(A)$ es el supremo de $\alpha A$, Primero
    probamos que es cota superior.

    Para cualquier $x \in A$, $x \leq sup(A)$, por tanto para cuaquier
    $y \in \alpha A$, como $y = \alpha x$ para alg\'un $x\in A$, tenemos
    (multiplicando ambos lados de la desigualdad, observando que $\alpha >0$)
    que $y = \alpha x \leq \alpha sup(A)$.

    Por otra parte, se verifica: 
    $$\forall \delta > 0, \: \exists a \in A \: \text{ tal que }
    \: sup(A)-a < \frac{\delta}{\alpha} $$

    Por tanto, dado un $y \in \alpha A$, como $y=\alpha x$ con $x \in A$,
    se satisface (multiplicando la desigualdad anterior por $\alpha$):
    
    $$\forall \delta > 0, \: \exists y \in \alpha A \: \text{ tal que }
    \: (\alpha sup(A))-y < \delta $$
    
    
  \item
    Análogo al caso anterior.
  \end{enumerate}
\end{enumerate}


\end{document}
