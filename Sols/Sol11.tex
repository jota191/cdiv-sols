\documentclass[10pt,a4paper]{article}
\usepackage[utf8x]{inputenc}
\usepackage{ucs}
\usepackage[english]{babel}
\usepackage{amsmath}
\usepackage{amsfonts}
\usepackage{amssymb}
\usepackage{makeidx}
\usepackage{graphicx}
\usepackage{lmodern}
\usepackage{kpfonts}
\usepackage{float}
\usepackage{enumitem}

\usepackage[left=2cm,right=2cm,top=2cm,bottom=2cm]{geometry}

\usepackage{titlesec}

\renewcommand{\o}{\circ}
\newcommand{\qed}{\blacksquare}
\newcommand{\R}{\mathbb{R}}

% elimina newline despues de \section:
\titleformat{\section}[runin]
{\normalfont\large\bfseries}{\thesubsection}{1em}{}      


\headsep17mm
\topmargin-1cm
\hoffset -1.5cm \voffset -1cm \textwidth 17cm \oddsidemargin
1.5cm \evensidemargin 1.5cm \textheight 22.5cm

\setlength\parindent{0cm}


\begin{document}

\vspace{0,3cm}

\begin{center}
{\bf \Large Resoluciones seleccionadas, semana 11}
\end{center}


\vspace{0,3cm}

\section*{1.6.}\emph{}

\noindent
Primero vamos a mostrar la continuidad en $0$.


$$
\displaystyle{\lim_{x\rightarrow 0} f(x)
  = \lim_{x\rightarrow 0} x^2 \sin \left(\frac{1}{x}\right)= 0\times
  \text{acotado} = f(0)}
$$


\noindent

Veamos que en $[0,a]$ la funci\'on no es mon\'otona.
Sabemos que dado $k \in \mathbb{N}$, $\sin(2\pi K)=0$, y
$\sin(2 \pi K + \frac{\pi}{2}) = 1$.
Por lo tanto si consideramos la funci\'on $\sin(\frac{1}{x})$, \'esta vale
$0$ en
$\displaystyle{\frac{1}{2\pi K}}$, y  $1$ en
$\displaystyle{\frac{1}{2 \pi K + \frac{\pi}{2}}}$.

\noindent
Dado $a \in \mathbb{R}$ podemos encontrar $K$ lo suficientemente grande
para que $\displaystyle{\frac{1}{2\pi K} \in [0,a]}$.

\noindent
N\'otese que
$$0 < \frac{1}{2\pi (K+1)} <
\frac{1}{2 \pi K + \frac{\pi}{2}} < \frac{1}{2\pi K} < a$$

\noindent
Pero entonces por el razonamiento anterior
$f$ vale 0 en $\displaystyle{\frac{1}{2\pi (K+1)}}$ y
$\displaystyle{\frac{1}{2\pi K}}$,
(los $x$ de m\'as a la izquierda y derecha) y
$\displaystyle{\left( \frac{1}{2 \pi K + \frac{\pi}{2}}\right)^2 > 0}$ en
$\displaystyle{\frac{1}{2 \pi K + \frac{\pi}{2}}}$ (el del medio).
Por lo tanto $f$ no es mon\'otona.

\noindent
Podemos buscar $K$ expl\'icitamente, quer\'iamos que
$ \displaystyle{\frac{1}{2\pi K}< a}$
por lo tanto, despejando necesitamos tomar $K$ entero tal que
$\displaystyle{ K >  \frac{1}{2\pi a}}$,
lo cual es posible nuevamente gracias a que los naturales no est\'an acotados.





\end{document}
