\documentclass[10pt,a4paper]{article}
\usepackage[utf8x]{inputenc}
\usepackage{ucs}
\usepackage[english]{babel}
\usepackage{amsmath}
\usepackage{amsfonts}
\usepackage{amssymb}
\usepackage{makeidx}
\usepackage{graphicx}
\usepackage{lmodern}
\usepackage{kpfonts}
\usepackage{float}
\usepackage{enumitem}

\usepackage[left=2cm,right=2cm,top=2cm,bottom=2cm]{geometry}

\usepackage{titlesec}

\renewcommand{\o}{\circ}
\newcommand{\qed}{\blacksquare}
\newcommand{\R}{\mathbb{R}}

% elimina newline despues de \section:
\titleformat{\section}[runin]
{\normalfont\large\bfseries}{\thesubsection}{1em}{}      


\headsep17mm
\topmargin-1cm
\hoffset -1.5cm \voffset -1cm \textwidth 17cm \oddsidemargin
1.5cm \evensidemargin 1.5cm \textheight 22.5cm




\begin{document}

\vspace{0,3cm}

\begin{center}
{\bf \Large Resoluciones seleccionadas, semana 05}
\end{center}


\vspace{0,3cm}

\section*{2.8}\emph{}%TODO escribir letra
\begin{enumerate}
\item
  Consideramos $K \in \R$, tal que $\forall x \in [a,b], K > |f(x)|$.
  Observemos que $K$ existe porque $f$ es integrable (ver te\'orico,
  ser acotada es una hip\'otesis requerida para una funci\'on ser
  integrable de Riemann).

  Entonces, asumiendo $y>x$ sin p\'erdida de generalidad:

  \begin{align*}
    K | y - x | & = \int_{x}^{y}{K \D{x}}  & &\\
    & \geq \int_{x}^{y}{|f(x)| \D{x}} & &\mbox{ \{ por comparaci\'on\}}\\
    & \geq \left| \int_{x}^{y}{f(x) \D{x}} \right|
    & &\mbox{ \{ propiedad de la integral \}}\\
    &= \left| \int_{0}^{y}{f(x) \D{x}} - \int_{0}^{x}{f(x) \D{x}} \right|
    & & \mbox{ \{ propiedad de la integral \}} \\
    &= \left|  F(x) - F(y) \right| & & \mbox{ \{ definici\'on de $F$ \}}
  \end{align*}

\item
  Consideramos el intervalo $[a_i, a_{i+1}]$.
  Tenemos que
  $K(a_{i+1} - a_i) = K|a_{i+1} - a_i| > |f(a_{i+1}) - f(a_i)| =
  |f(a_i)) - f(a_{i+1})|$. (La desigualdad vale por la hip\'otesis de que $f$
  es Lipschitz).

  Por otra parte $f(a_i) - inf(f,[a_i, a_{i+1}])>0$
  por definici\'on de \'infimo.

  Aplicando la desigualdadd triangular,
  $|f(a_i) - f(a_{i+1})| + |f(a_i) - inf(f,[a_i, a_{i+1}])|
  > | f(a_i)- inf(f,[a_i, a_{i+1}])| = f(a_i)- inf(f,[a_i, a_{i+1}])$.

  Por tanto, por transitividad
  $K(a_{i+1} - a_i) > f(a_i)- inf(f,[a_i, a_{i+1}])$. Reordenando obtenemos lo
  pedido. Para la segunda inecuaci\'on el procdimiento es an\'alogo.

  Observemos que sumando miembro a miembro las dos inecuaciones
  se puede concluir que:
  $(sup(f,[a_i, a_{i+1}]) - inf(f,[a_i, a_{i+1}]) \leq 2K (a_{i+1} -a_i)$.
  
  Vamos a probar que $f: [a,b] \rightarrow \R$ lipschiziana es integrable.

  Como se vi\'o en el te\'orico, una condici\'on necesaria para $f$ integrable
  es que existan particiones tales que las sumas superiores e inferiores
  difieran en menos de $\epsilon$, para cualquier $\epsilon>0$.
  
  Consideramos una partici\'on equiespaciada $P_n$, del intervalo $[a,b]$.
  Tenemos:
  \begin{align*}
    S^{*}(f,P_n) - S_{*}(f,P_n) &
    = \sum_{i=0}^{i-1}{sup(f,[a_i, a_{i+1}]) (a_{i+1} -a_{1})}
    - \sum_{i=0}^{i-1}{inf(f,[a_i, a_{i+1}]) (a_{i+1} -a_{1})}& &\\
    &= \sum_{i=0}^{i-1}{(sup(f,[a_i, a_{i+1}]) - inf(f,[a_i, a_{i+1}]))
      (a_{i+1} -a_{1})}& &\\
    &= \sum_{i=0}^{i-1}{2K \frac{b-a}{n} \frac{b-a}{n}}& &\\
    &= 2Kn \frac{b-a}{n} \frac{b-a}{n}& &\\
    &= \frac{2K (b-a)^2}{n} & &\\
  \end{align*}
  
  Entonces, dado $\epsilon$ alcanza con tomar $n$ tal que
  $ n > \frac{2K (b-a)^2}{\epsilon}$.
\item
  Supongamos lo contrario, consideramos $x_0 \in [a,b]$ tal que $f(x_0)>0$.
\end{enumerate}

  
\end{document}
