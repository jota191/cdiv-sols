\documentclass[10pt,a4paper]{article}
\usepackage[utf8x]{inputenc}
\usepackage{ucs}
\usepackage[english]{babel}
\usepackage{amsmath}
\usepackage{amsfonts}
\usepackage{amssymb}
\usepackage{makeidx}
\usepackage{graphicx}
\usepackage{lmodern}
\usepackage{kpfonts}
\usepackage{float}
\usepackage{enumitem}

\usepackage[left=2cm,right=2cm,top=2cm,bottom=2cm]{geometry}

\usepackage{titlesec}

\renewcommand{\o}{\circ}
\newcommand{\qed}{\blacksquare}
\newcommand{\R}{\mathbb{R}}

% elimina newline despues de \section:
\titleformat{\section}[runin]
{\normalfont\large\bfseries}{\thesubsection}{1em}{}      


\headsep17mm
\topmargin-1cm
\hoffset -1.5cm \voffset -1cm \textwidth 17cm \oddsidemargin
1.5cm \evensidemargin 1.5cm \textheight 22.5cm




\begin{document}

\vspace{0,3cm}

\begin{center}
{\bf \Large Resoluciones seleccionadas, semana 10}
\end{center}


\vspace{0,3cm}

\section*{1.6.d}\emph{}

\noindent
Para este ejercicio podr\'iamos considerar funciones
trigonom\'etricas, que intuitivamente ya manejamos.
Otro ejemplo no trivial es definir una funci\'on como la de [REF],
$\{ \} : \mathbb{R}\rightarrow\mathbb{R}$ donde $\{x\}$ es la distancia
de $x$ al entero mas pr\'oximo.
Es f\'acil probar que esta funci\'on est\'a acotada
($0\leq\{x\}\leq\frac{1}{2}$). y ver que intuitvamente no tiene l\'imite
infinito. Formalicemos \'esto. Lo haremos en particular para $+\infty$
pero el otro caso es an\'alogo

\noindent
Observemos que la definici\'on de $\displaystyle{\lim_{x\rightarrow \infty}
  \{x\}= l}$.
es que dado $\epsilon > 0$ podemos encontrar
$K\in\mathbb{R}$ tal que si $x > K$ entonces $|x-l|<\epsilon$.
M\'as precisamente:
$$
\forall \epsilon>0, \exists K\in\mathbb{R} \: : \: \forall x>K, \: |\{x\}-l|<
\epsilon
$$

\noindent
La {\bf negaci\'on} de esta proposici\'on es 

$$
\exists \epsilon>0, \forall K\in\mathbb{R} / \exists x>K |\{x\}-l|>\epsilon
$$

\noindent
Es decir, podemos dar un $\epsilon$ tal que sin importar que tan grande sea
$K$, podemos encontrar $x>K$ con $|\{x\}-l|>\epsilon$.

\noindent
Tomamos $\epsilon=\frac{1}{4}$. Dado $K$, elegimos $x_0\in\mathbb{N}$ con
$x_0>K$ (Podemos porque ya demostramos que los naturales no est\'an acotados).

\noindent
Tomamos $x_1 = x_0 + \frac{1}{2}$.
Por la desigualdad trialgular:

$$
|\{x_1\} - l| + |l - \{x_2\}| \geq |\{x_1\} - \{x_2\}| = \frac{1}{2}
$$

\noindent
Pero entonces tenemos dos reales positivos que suman $\frac{1}{2}$, por lo que
uno de ellos debe valer por lo menos $\frac{1}{4}$, como quer\'iamos.



\end{document}
