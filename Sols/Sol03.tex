\documentclass[10pt,a4paper]{article}
\usepackage[utf8x]{inputenc}
\usepackage{ucs}
\usepackage[english]{babel}
\usepackage{amsmath}
\usepackage{amsfonts}
\usepackage{amssymb}
\usepackage{makeidx}
\usepackage{graphicx}
\usepackage{lmodern}
\usepackage{kpfonts}
\usepackage{float}

\usepackage[left=2cm,right=2cm,top=2cm,bottom=2cm]{geometry}

\usepackage{titlesec}

\renewcommand{\o}{\circ}
\newcommand{\qed}{\blacksquare}

% elimina newline despues de \section:
\titleformat{\section}[runin]
{\normalfont\large\bfseries}{\thesubsection}{1em}{}      


\headsep17mm
\topmargin-1cm
\hoffset -1.5cm \voffset -1cm \textwidth 17cm \oddsidemargin
1.5cm \evensidemargin 1.5cm \textheight 22.5cm




\begin{document}

\vspace{0,3cm}

\begin{center}
{\bf \Large Resoluciones seleccionadas, semana 03}
\end{center}


\vspace{0,3cm}

\section*{1.1.f}\emph{Probar $\sum_{i=1}^n i(i!)= (n + 1)! - 1$}

\noindent
\emph{Solución:}

\begin{itemize}
\item Paso base:
  
  Con $n=1$ tenemos $\sum_{i=1}^1 i(i!)=1=(1+1)! -1$.
  \footnote{Nótese que tomamos $1$ como base para no tener el caso
    de la suma degenerada, pero podíamos comenzar por $0$ perfectamente.
  En general $\sum_{i=1}^0 f(i) = 0$}

\item Paso inductivo

  \begin{itemize}
  \item HI: Para $h\in \mathbb{N}$, se satisface
    $\sum_{i=1}^h i(i!)= (h + 1)! - 1$
  \item TI: Deducimos $\sum_{i=1}^{h+1} i(i!)= ((h + 1) + 1)! - 1$
  \end{itemize}
\emph{Demostración:}

\begin{align*}
  \sum_{i=1}^{h+1} i(i!)
  &= \sum_{i=1}^{h} i(i!) + (h+1)(h+1)! &\\
  &= (h+1)! - 1 + (h+1)(h+1)! {\text{ (por hipótests) } } &\\
  &= (h+1)! + (h+1)(h+1)! - 1 &\\
  &= (h+2)(h+1)! - 1 &\\
  &= (h+2)! - 1 &\\
  &= ((h + 1) + 1)! - 1  \qed &\\
\end{align*}
  
\end{itemize}





\section*{1.2.a}
\emph{Probar $(1+x)^n \geq 1+nx \quad \forall \, x>-1,
  \forall \, n \in \mathbb{N}$ }

\noindent
\emph{Solución:}

\begin{itemize}
\item Paso base:
  
  Con $n=0$ tenemos trivialmente $(1+x)^0 \geq 1+0x$
  (ambas expresiones valen $1$).
  
\item Paso inductivo

  \begin{itemize}
  \item HI: Para $h \in \mathbb{N}$ se cumple
    $(1+x)^h \geq 1+hx \quad \forall \, x>-1$
  \item TI: $(1+x)^{h+1} \geq 1+(h+1)x \quad \forall \, x>-1$
  \end{itemize}
\emph{Demostración:}

\begin{align*}
  (1+x)^{h+1} &= (1+x)^{h+1} &\\
  &=    (1+x)^{h}(1+x) &\\
  &\geq (1+hx)(1+x) \qquad \text{ (por hipótesis, y porque $1+x>0$)}&\\
  &=    1+hx+x+hx^2 &\\
  &=    1+(h+1)x+hx^2 &\\
  &\geq 1+(h+1)x \qed \qquad \text{(porque $hx^2>0$)} &\\
\end{align*}
  
\end{itemize}


\end{document}
